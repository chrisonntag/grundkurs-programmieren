\begin{frame}{Syntax in Python}
\begin{itemize}
    \item der Code wird durch Einrückungen (Tabs) strukturiert (vgl. \texttt{for}-Schleife)
    \item runde Klammern () sind meist für Parameter (\lstinline{print("Hier der Text")}) 
    \item eckige Klammern [] sind meist für `listenartige' Datenstrukturen (Arrays, Listen in Python)
    \item Leerzeichen und Zeilenumbrüche sind meist optional, verbessern aber die Lesbarkeit des Programms
    \item Groß- und Kleinschreibung muss meist beachtet werden
\end{itemize}
\end{frame}

\begin{frame}{Konventionen für lesbareren Code}
Damit Code einheitlich gut lesbar ist, gibt es für Programmiersprachen 
`Coding Conventions', die zwar nicht erfüllt werden \textit{müssen}, aber 
einen guten Eindruck hinterlassen und zur besseren Lesbarkeit beitragen. \\
Auszug (\textit{PEP8}): 
\begin{itemize}
    \item Variablen- und Funktionsnamen klein und wenn nötig \\ mit \_ schreiben 
    \item Eine Einrückungsebene in Python entspricht genau 4 Leerzeichen (keine Tabulatorzeichen) 
    \item Am Anfang jeder Python-Datei steht ein Doc-String (Kommentar), der kurz den Inhalt der Datei bescheibt
    \item \dots bei Funktionen auch
\end{itemize}
\end{frame}

