\begin{frame}[fragile]{Typen und Variablen}
Wir haben ganz unbewusst bereits zwei Datentpen benutzt
\begin{itemize}
    \item \texttt{String, str}: 
    \begin{lstlisting}
print("Ich bin vom Typ String, eine Reihe von Zeichen")
    \end{lstlisting}
    \item \texttt{Integer, int}: 
    \begin{lstlisting}
sleep(3)   # 3 ist ein Integer
    \end{lstlisting}
\end{itemize}
\end{frame}

\begin{frame}[fragile]{Typen und Variablen}
\begin{itemize}
    \item Wahrheitswert (Boolean, bool): 
    \begin{lstlisting}
wahr = True
falsch = False
    \end{lstlisting}
    \item Gleitkommazahlen (Float, float):
    \begin{lstlisting}
pi = 3.1415
    \end{lstlisting}
    \item In Python gibt es sogar komplexe Zahlen \texttt{complex}:
    \begin{lstlisting}
a = complex('3+5j')
    \end{lstlisting}

\end{itemize}
\end{frame}

\begin{frame}[fragile]{Typen und Variablen}
\begin{itemize}
    \item Zuweisung von Variablen mit dem Zuweisungsoperator \texttt{=}
    \begin{lstlisting}
a = 5
b = 3.14
c = "Hallo Grundkurs:Programmieren"
    \end{lstlisting}
    \item der Variable kann auch das Ergebnis einer Operation zugewiesen werden
    \begin{lstlisting}
a = 1000
b = 200
percent = b / a * 100
    \end{lstlisting}
    \item mehrfache Zuweisung 
    \begin{lstlisting}
a, b, c = 5, 3.14, "Hallo Grundkurs:Programmieren"
x = y = z = 42
    \end{lstlisting}
\end{itemize}
\end{frame}


\begin{frame}[fragile]{Typsicherheit}
 \begin{itemize}
    \item streng-getypte Sprachen: \textbf{Java}, C/C++, \dots
    \begin{lstlisting}
int x = 42
float pi = 3.14
String greeting = "Hallo Grundkurs:Programmieren"
boolean truth = true
    \end{lstlisting}
    \item dynamisch (schwach) getypte Sprachen: \textbf{Python}, Javascript
    \begin{lstlisting}
x = 42
pi = 3.14
greeting = "Hey there"
truth = True
    \end{lstlisting}
\end{itemize}   
\end{frame}

\begin{frame}[fragile]{Typsicherheit}
\begin{block}{Aufgabe}
Gib folgende Ausdrücke in den Python Interpreter ein:
\begin{lstlisting}
>>> 3 + 3.14
>>> "Mein Alter: " + 5
>>> True + 1
\end{lstlisting}
\end{block}
\pause{}
\begin{exampleblock}{True + 1 == 2?}
Intern werden die Keywords \lstinline{True} und \lstinline{False} auf die 
Werte \texttt{1} und \texttt{0} vom Typ \texttt{int} abgebildet.
\end{exampleblock}

\end{frame}


\begin{frame}[fragile]{Typen und Variablen}
    Die \lstinline{input()} Funktion nimmt Benutzereingaben auf der Kommandozeile 
    entgegen und gibt sie zurück. Du brauchst kein zusätzliches Modul importieren.
    \begin{block}{Aufgabe}
       Lasse dich von deinem Programm begrüßen, indem du mit \texttt{input} deine 
       Eingabe in einer Variable speicherst.
    \end{block}
    \pause{}
    \begin{exampleblock}{Lösung}
        \begin{lstlisting}
>>> name = input()
'Christoph'
>>> print("Hallo " + name)
        \end{lstlisting}
    \end{exampleblock}
\end{frame}

\begin{frame}[fragile]{Typconversion}
    \begin{alertblock}{Achtung}
    Die \lstinline{input()} Funktion interpretiert jede Benutzereingabe
    als \texttt{String}. Wenn man Zahlen aufnehmen will, muss der Typ
    `gecastet' werden, d.h. `umgewandelt'.
    \begin{itemize}
        \item \lstinline{int()}: Castet zu int.
        \item \lstinline{str()}: Castet zu String.
    \end{itemize}
    Was passiert bei \lstinline{int("HalloWelt")}?
    \pause{}
    Ein \texttt{ValueError} wird geworfen. (Exception handling nicht 
    in diesem Kurs)
    \end{alertblock}
\end{frame}

