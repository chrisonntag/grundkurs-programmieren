\begin{frame}[fragile]{Die \texttt{for}-Schleife}
\begin{block}{Aufgabe}
    Schreibe ein Programm, dass von 1 bis 100 zählt.
    \begin{lstlisting}
print("Ich kann uebrigens bis 100 zaehlen ... ;-)")
...
    \end{lstlisting}
\end{block}
\pause{}
\begin{exampleblock}{Lösungsvariante 1}
    \begin{lstlisting}
print(1)
print(2)
print(3)
print(4)
print(5)
print(6)
print(7)
...
    \end{lstlisting}
\end{exampleblock}
\end{frame}

\begin{frame}[fragile]{Die \texttt{for}-Schleife}
Vereinfachte weniger schreibintensive Variante:
\begin{exampleblock}{Lösungsvariante 2}
    \begin{lstlisting}
for i in range(1, 100):
    print(i)
    \end{lstlisting}
\end{exampleblock}
\begin{itemize}
    \item gibt die Werte von 1 bis 99 aus
    \pause{} 
    \item gib in den Interpreter \texttt{help()} ein, um genaueres über 
    eine Funktion zu erfahren. 
    \pause{}
    \item aus der Dokumentation: \texttt{range(i, j) produces i, i+1, i+2, ..., j-1.}
\end{itemize}
\end{frame}

\begin{frame}[fragile]{Die \texttt{for}-Schleife}
\begin{itemize}
    \item Andere Sprachen erzeugen kein 'extra Objekt', sondern zählen eine 
    Variable hoch
    \item genauer: setzt $i=1$, addiert in jedem Schritt 1 auf, 
    solange $i < 100$ ist. 
\end{itemize}
\begin{lstlisting}[language=Java]
for(int i=1, i < 100, i++) {
    // gib i aus
}
\end{lstlisting}
\end{frame}

\begin{frame}[fragile]{Die \texttt{for}-Schleife}
\begin{block}{Aufgabe}
Ändere das Programm von gerade eben so, dass wirklich bis 100 gezählt wird.
\end{block}
\pause{}
\begin{exampleblock}{Lösung}
\begin{lstlisting}
for in in range(1, 101):
    print(i)
\end{lstlisting}
\end{exampleblock}
\end{frame}

\begin{frame}[fragile]{Die \texttt{for}-Schleife}
\begin{block}{Aufgabe}
Bildet 2-3er Gruppen und schreibt ein Programm, dass von 10 bis 0 herunterzählt.
\textit{Hinweis:} Verwendet zum pausieren zwischen den Zahlen wieder \texttt{sleep()}.
\lstinline{from time import sleep} nicht vergessen.
\end{block}
\pause{}
\begin{exampleblock}{Lösung}
    siehe Beamer
\end{exampleblock}
\end{frame}

