\begin{frame}{Exkurs: Projekt Dimensionen}
\begin{itemize}
    \item Programme in der Regel wesentlich länger als 100 Zeilen
    \pause{}
    \item Metrik zur Aufwandsberechnung z.B. Lines of Code. (LOC, SLOC)
        \begin{itemize}
            \item Windows XP: ca. 40 Mio. SLOC
            \item Mac OS X Tiger: ca. 86 Mio. SLOC
        \end{itemize}
    \pause{}
    \item nicht zwangsläufig Rückschlüsse auf die Komplexität oder Funktionalität
\end{itemize}
\end{frame}

\iffalse
\begin{frame}{Exkurs: Projekt Dimensionen}
\begin{itemize}
    \item Projektgröße: Als Maßstab gilt meist das Budget. „Kleine“ Projekte sind oft
auch einfach und flach strukturiert. Hier sollte minimaler Aufwand betrieben
werden.
    \pause{}
    \item Projektrisiken: Hohe Risiken verlangen nach sorgfältigem Risikomanagement.
    \pause{}
    \item Komplexität: Hohe Komplexität stellt Anforderungen an die Dokumentation,
Planung und Kontrolle. Man sollte prüfen, ob eine Aufteilung des Projektes
möglich ist.
    \pause{}
    \item Projektdauer: „Langläufer“ erfordern gute Dokumentation und eine laufende
Überprüfung des zugrunde liegenden „Business Case“.
\end{itemize}
\end{frame}

\begin{frame}{Exkurs: Projekt Dimensionen}
\begin{itemize}
    \item Bedeutung für das Unternehmen: Eine hohe Bedeutung ist verbunden mit besonderen Informationspflichten in das Management und hohen Erwartungen
bzgl. der Termineinhaltung.
    \pause{}
    \item Auftraggeber (intern / extern): Externe Kundenprojekte haben hohe Anforderungen
an die Leistungsdokumentation, ohne die z.B. keine Abrechnung oder
Leistungsabnahme möglich ist.
    \pause{}
    \item Innovationsgrad: „Standardprojekte“ sind etabliert und werden nach einfachen
Standards abgewickelt.
    \pause{}
    \item Projektphasen: Eine solide Projektdefinition vor dem Projektstart erspart hohen
Aufwand im Projektablauf. Je nach Projektphase sind unterschiedliche Abteilungen
und Mitarbeiter in dem Projekt aktiv. Kritisch sind dabei die Übergaben.
\end{itemize}
\end{frame}
\fi
