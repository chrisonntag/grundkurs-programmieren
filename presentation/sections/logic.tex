\begin{frame}[fragile]{logische Operatoren}
Den Zuweisungsoperator \texttt{=} haben wir unbewusst bereits kennengelernt. 
Er weißt einem Wert bspws.\ einen Namen zu (\lstinline{x = 42}). Die Operatoren 
\texttt{+}, \texttt{-}, \texttt{/} oder \texttt{*} verwenden wir intuitiv.\\

Zum besseren Vergleichen brauchen wir weitere Operatoren:
\begin{itemize}
    \item \texttt{==}: prüft zwei Werte auf Gleichheit
    \item \texttt{!=}: prüft zwei Werte auf Ungleichheit
    \item \texttt{>}: größer
    \item \texttt{<}: kleiner (vgl.\ for-Schleife)
    \item kleiner-gleich?
    \pause{}
    \item \texttt{<=, >=}, kleiner-gleich, größer-gleich
    \item \texttt{and}: logisches `Und'
    \item \texttt{or}: logisches `Oder'
    \item \texttt{not}: verneint einen Ausdruck
\end{itemize}
\end{frame}

\begin{frame}[fragile]{logische Operatoren}
Was ergeben folgende Ausdrücke? Überprüfe sie mit dem Python Interpreter.

\begin{lstlisting} 
3 > 4 
\end{lstlisting}

\begin{lstlisting} 
6 != 7
\end{lstlisting}

\begin{lstlisting} 
"Hallo" < "Hallo Welt!"
\end{lstlisting}

\begin{lstlisting} 
"Hallo" == "Hallo Welt"
\end{lstlisting}

\begin{lstlisting} 
"Hallo Welt" == "Hallo" and 3 > 4
\end{lstlisting}

\begin{lstlisting} 
not not 5 == 5
\end{lstlisting}

\end{frame}
