\begin{frame}[fragile]{Die \texttt{while}-Schleife}
Ähnlich zur \texttt{for}-Schleife mit dem Unterschied, dass nicht `durch eine Liste gelaufen' wird, 
sondern die Schleife immer wiederholt wird, solange die Bedingung wahr ist.

    \begin{lstlisting}
while Bedingung:
    # Programmcode

# ausserhalb der Schleife
    \end{lstlisting}
\end{frame}

\begin{frame}{Die \texttt{while}-Schleife}
\begin{block}{Aufgabe}
Schreibe ein Programm, welches alle Zahlen von 1 bis 100 auf den Bildschirm schreibt, ohne dafür eine 
for Schleife zu verwenden.
\end{block}
\end{frame}


%\begin{frame}{Die \texttt{while}-Schleife}
%\begin{block}{Aufgabe}
%Die Folge aus Fibonacci-Zahlen wird wie folgt gebildet:
%\begin{itemize}
%    \item Das erste und das zweite Element sind 0 und 1.
%    \item Jedes folgende Element wird gebildet, in dem die letzten zwei Elemente addiert werden.
%\end{itemize}
%Das heisst, die Folge sieht so aus: 0,1,1,2,3,5,8,13,21,34,\dots
%
%Schreibe ein Programm, welches die Fibonacci-Zahlen bis zu einer vom Benutzer gewählten Zahl ausgibt.
%\end{block}
%\end{frame}

%\begin{frame}[fragile]{while-Schleife}
%\begin{exampleblock}{Lösung 1}
%\begin{lstlisting}
%num = 10
%# num = int(input("Bis zu welcher Zahl? "))
%fib1 = 0
%fib2 = 1
%count = 0
%
%if num <= 0:
%   print("Bitte eine positive Zahl eingeben")
%elif num == 1:
%   print("Fibonacci bis zur " + num + "ten Zahl:")
%   print(fib1)
%else:
%    print("Fibonacci bis zur " + num + "ten Zahl:")
%    while count < num:
%	    print(fib1)
%	    fibn = fib1 + fib2
%	    fib1 = fib2
%	    fib2 = fibn
%	    count = count + 1
%        # count += 1
%\end{lstlisting}
%\end{exampleblock}
%\end{frame}

%\begin{frame}[fragile]{while-Schleife}
%\begin{exampleblock}{Lösung 2}
%\begin{lstlisting}
%def fibonacci(n):
%    a = 0
%    b = 1
%    for i in range(0, n):
%        temp = a
%        a = b
%        b = temp + b
%    return a
%\end{lstlisting}
%\end{exampleblock}
%\end{frame}

\begin{frame}{Die \texttt{while}-Schleife}
\begin{block}{Aufgabe: Das kleine 1$\times$1}
Schreibe ein Programm, das den Benutzer `zufällige' 1$\times$1 Rechenaufgaben abfragt und 
die Lösung überprüft. (\lstinline{from random import randint}). Nach jeder Rechenaufgabe 
soll der Benutzer gefragt werden, ob noch eine Aufgabe gestellt werden soll.
\end{block}
\end{frame}

\begin{frame}[fragile]{Die \texttt{while}-Schleife}
\begin{exampleblock}{Lösung}
\begin{lstlisting}
from random import randint

print("Herzlich willkommen zum kleinen EinmalEins Rechner!")
weiter = "Ja"

while weiter == "Ja":
    zahl1 = randint(1, 9)
    zahl2 = randint(1, 9)
    print(str(zahl1) + " x " + str(zahl2))

    loesung = int(input("= "))
    if loesung == zahl1 * zahl2:
        print("Korrekt! Glueckwunsch")
    else:
        print("Leider nicht ganz korrekt. Richtig waere: " + str(zahl1 * zahl2))
    weiter = input("Weiter? (Ja/Nein): ")

print("Vielen Dank!")
\end{lstlisting}
\end{exampleblock}
\end{frame}


