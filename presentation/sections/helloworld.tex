\begin{frame}[fragile]{Hello World}
    \begin{lstlisting}
    print("Hello World!")
    \end{lstlisting}
    \begin{itemize}
        \item IDLE suchen und starten
        \item eintippen und mit Enter ausführen
        \item gibt den Text (String) `Hello World!' auf der Konsole aus
    \end{itemize}
    \begin{exampleblock}{Glückwunsch}
    Ihr habt gerade euer erstes Code-Fragment geschrieben und ausgeführt!
    \end{exampleblock}
\end{frame}

\begin{frame}[fragile]{Zeichnen mit \texttt{turtle}}
    \begin{lstlisting}
import turtle

turtle.shape("turtle") # str bzw. String
turtle.forward(100) # int bzw. Integer
turtle.left(45)
turtle.forward(100)
    \end{lstlisting}
    \begin{itemize}
        \item bei jedem Befehl wird eine Funktion aufgerufen
        \item \texttt{turtle} nimmt Befehle entgegen (forward, left, right) mit sog. 
        Eingangsparametern (die wiederum von einem bestimmten Typ sind)
    \end{itemize}
\end{frame}

\begin{frame}[fragile]{Zeichnen mit \texttt{turtle}}
    \begin{block}{Aufgabe 1}
        Schreibe ein Programm, dass ein Quadrat der Seitenlänge 100 zeichnet.
    \end{block}
    \pause{}
    \begin{exampleblock}{Lösung}
        \begin{lstlisting}
import turtle

turtle.forward(100)
turtle.left(90)
turtle.forward(100)
turtle.left(90)
turtle.forward(100)
turtle.left(90)
turtle.forward(100)
        \end{lstlisting}
    \end{exampleblock}
\end{frame}

