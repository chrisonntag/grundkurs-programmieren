\begin{frame}[fragile]{Hello World}
    \begin{lstlisting}
    print("Hello World!")
    \end{lstlisting}
    \begin{itemize}
        \item IDLE suchen und starten
        \item eintippen und mit Enter ausführen
        \item gibt den Text (String) `Hello World!' auf der Konsole aus
    \end{itemize}
    \begin{exampleblock}{Glückwunsch}
    Ihr habt gerade euer erstes Code-Fragment geschrieben und ausgeführt!
    \end{exampleblock}
\end{frame}

\begin{frame}[fragile]{Hello World}
    \begin{block}{Aufgabe 1}
        Erweitere das Programm so, dass der \texttt{String} `Hello World'
        6-mal auf der Konsole ausgegeben wird.
    \end{block}
    \pause{}
    \begin{exampleblock}{Lösung}
        \begin{lstlisting}
print("Hello World")
print("Hello World")
print("Hello World")
print("Hello World")
print("Hello World")
print("Hello World")
        \end{lstlisting}
    \end{exampleblock}
\end{frame}

\begin{frame}[fragile]{Hello World}
    \begin{itemize}
        \item \texttt{sleep(num)}
            \begin{itemize}
                \item stoppt die Ausführung des Programms für \texttt{num} Sekunden
                \item ist nicht Teil des Standard-Pythons, sondern muss importiert 
                    werden. \lstinline[columns=fixed]{from time import sleep}
            \end{itemize}
    \end{itemize}
    \pause{}
    \begin{block}{Aufgabe 2}
        Lass das Programm einen realistischen Monolog ausführen. 
        Verwende hierzu \texttt{sleep}
    \end{block}
    \pause{}
    \begin{exampleblock}{Lösung}
        \begin{lstlisting}
from time import sleep

print("Hey, it's James!")
sleep(2)
print("I'm working on a ChatBot right now")
sleep(3)
...
        \end{lstlisting}
    \end{exampleblock}

\end{frame}

