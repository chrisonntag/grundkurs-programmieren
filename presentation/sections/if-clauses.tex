\begin{frame}[fragile]{Bedingte Ausführung}
Hilft uns beim strukturieren des Programms in verschiedene Richtungen.
Struktur:

\begin{lstlisting}
if Bedingung:
    # Programmcode
else:
    # alternativer Programmcode, wenn Bedingung nicht zutrifft
\end{lstlisting}
\pause{}
\begin{lstlisting}
zahl = input()

if zahl > 10:
    print("Die Zahl ist > 10.")
else:
    print("Die Zahl ist < 10.")
\end{lstlisting}
\end{frame}

\begin{frame}[fragile]{Bedingte Ausführung}
    \begin{block}{Aufgabe}
        Schreibe eine Programm, das zwei Zahlen vom Benutzer einliest, testet, 
        ob die erste durch die zweite Zahl teilbar ist und das Ergebnis ausgibt.\\
        \pause{}
        \textit{Hinweis:}\\
        Eine Zahl ist durch eine andere teilbar, wenn der Rest bei der Division beider 
        Zahlen 0 ergibt. Die Modulo-Operation \texttt{a \% b} gibt uns den Rest zweier Zahlen, wenn man 
        sie teilen würde.
    \end{block}
\end{frame}

\begin{frame}[fragile]{Bedingte Ausführung}
    \begin{exampleblock}{Lösung}
        \begin{lstlisting}
zahl1 = int(input("Erste Zahl: "))
zahl2 = int(input("Zweite Zahl: "))

if zahl1 % zahl2 == 0:
    print("Zahl 1 ist durch die Zweite teilbar!")
else: 
    print("Leider kann man die Zahlen nicht ohne Rest teilen")
        \end{lstlisting}
    \end{exampleblock}
\end{frame}

\begin{frame}[fragile]{Bedingte Ausführung}
Mehrere Bedingungen?
\pause{}
\begin{lstlisting}
zahl = int(input())

if zahl > 10:
    print("Die Zahl ist > 10.")
elif zahl > 5:
    print("Die Zahl ist > 5.")
else:
    print("Die Zahl ist < 5.")

\end{lstlisting}
\end{frame}

\begin{frame}{Bedingte Ausführung}
    \begin{block}{Aufgabe}
        Schreibe ein Programm, dass ein eingegebenes Hundealter in Menschenjahre umrechnet.
        \begin{itemize}
            \item 1 Hundejahr $=$ 14 Jahre
            \item 2 Hundejahre $=$ 22 Jahre
            \item Über 2 Jahren $= 22 + (Alter -2) * 5$
        \end{itemize}
    \end{block}
\end{frame}

\begin{frame}[fragile]{Bedingte Ausführung}
    \begin{exampleblock}{Lösung}
        \begin{lstlisting}
age = int(input("Alter des Hundes: "))
if age < 0:
    print("Das stimmt wohl kaum!")
elif age == 1:
    print("entspricht ca. 14 Jahren")
elif age == 2:
    print("entspricht ca. 22 Jahren")
elif age > 2:
    human = 22 + (age -2)*5
    print("entspricht ca. " + str(human) + "Jahren")
        \end{lstlisting}
    \end{exampleblock}
\end{frame}


\begin{frame}{Bedingte Ausführung \hyphen{} Schachteln von \texttt{if}-Abfragen}
    \begin{block}{Aufgabe}
        Schreibe ein Programm, das errechnet, ob ein eingegebenes Jahr ein
        Schaltjahr ist, also ob der 29. Februar existiert oder nicht.
        \begin{itemize}
            \item Zusatztag existiert alle 4 Jahre
            \item alle 100 Jahre wird jedoch darauf verzichtet
            \item alle 400 Jahre weicht man von der Verzichtregel ab
        \end{itemize}
    \end{block}
    \pause{}
    \begin{exampleblock}{gemeinsame Lösung}
    siehe Beamer
    \end{exampleblock}
\end{frame}

\begin{frame}[fragile]{Bedingte Ausführung \hyphen{} Schachteln von \texttt{if}-Abfragen}
    \begin{exampleblock}{Lösung}
        \begin{lstlisting}
year = int(input())

if (year % 4) == 0:
   if (year % 100) == 0:
       if (year % 400) == 0:
           print("ist ein Schaltjahr")
       else:
           print("ist kein Schaltjahr")
   else:
       print("ist ein Schaltjahr")
else:
   print("ist kein Schaltjahr")
       \end{lstlisting}
    \end{exampleblock}
\end{frame}


